% YMH429 - Yazılım Kalite Güvencesi ve Testi | Final Raporu
% E-Ticaret Sipariş ve Ödeme Servisleri için REST API Test Otomasyon Çatısı
% Recep Öztürk - 22290380

\documentclass[12pt,a4paper]{article}

\usepackage[utf8]{inputenc}
\usepackage[T1]{fontenc}
\usepackage{times}
\usepackage[margin=2.5cm]{geometry}
\usepackage{setspace}
\usepackage{graphicx}
\usepackage{float}
\usepackage{hyperref}
\usepackage{listings}
\usepackage{xcolor}
\usepackage{booktabs}
\usepackage{array}

\definecolor{codegray}{rgb}{0.5,0.5,0.5}
\definecolor{backcolour}{rgb}{0.95,0.95,0.92}

\lstdefinestyle{mystyle}{
    backgroundcolor=\color{backcolour},
    basicstyle=\ttfamily\footnotesize,
    breaklines=true,
    captionpos=b,
    numbers=left,
    numbersep=5pt,
    numberstyle=\tiny\color{codegray},
    frame=single
}
\lstset{style=mystyle}

\onehalfspacing

\title{
    \vspace{-2cm}
    \textbf{ANKARA ÜNİVERSİTESİ} \\
    \textbf{MÜHENDİSLİK FAKÜLTESİ} \\
    \textbf{YAZILIM MÜHENDİSLİĞİ BÖLÜMÜ} \\
    \vspace{1cm}
    \large YMH429 -- YAZILIM KALİTE GÜVENCESİ VE TESTİ \\
    2024-2025 GÜZ DÖNEMİ \\
    \vspace{1cm}
    \textbf{DÖNEM PROJESİ} \\
    \vspace{0.5cm}
    \Large E-Ticaret Sipariş ve Ödeme Servisleri için \\
    REST API Test Otomasyon Çatısı Geliştirilmesi
}
\author{}
\date{}

\begin{document}

\maketitle
\thispagestyle{empty}

\vspace{2cm}
\begin{center}
\begin{tabular}{|c|c|}
\hline
\textbf{Öğrenci Numarası} & \textbf{Adı-Soyadı} \\
\hline
22290380 & Recep Öztürk \\
\hline
\end{tabular}
\end{center}

\vfill
\begin{center}
\textbf{Aralık 2024}
\end{center}

\newpage
\tableofcontents
\newpage

%%%%%%%%%%%%%%%%%%%%%%%%%%%%%%%%%%%%%%%%%%%%%%%%%%%%%%%%%%%%%%%%%%%%%%%%%%%%%%%
% BÖLÜM 1: GİRİŞ VE İLGİLİ ÇALIŞMALAR (15 puan)
%%%%%%%%%%%%%%%%%%%%%%%%%%%%%%%%%%%%%%%%%%%%%%%%%%%%%%%%%%%%%%%%%%%%%%%%%%%%%%%
\section{Giriş ve İlgili Çalışmalar}

\subsection{Projenin Amacı ve Önemi}

Günümüz yazılım sistemlerinde uygulama programlama arayüzleri (API) kritik bir rol üstlenmektedir. Özellikle e-ticaret, finans ve sağlık gibi sektörlerde, uygulamalar arası veri alışverişi büyük ölçüde RESTful API'ler üzerinden gerçekleştirilmektedir. Bu API'lerin güvenilir ve hatasız çalışması, hem kullanıcı deneyimi hem de iş sürekliliği açısından hayati önem taşımaktadır.

API'lerde yaşanan hatalar ciddi sonuçlara yol açabilmektedir:
\begin{itemize}
    \item \textbf{Finansal kayıplar:} Hatalı işlemler, yanlış fiyatlandırma veya ödeme hataları doğrudan gelir kaybına neden olabilir.
    \item \textbf{İtibar zararı:} Sürekli hata veren sistemler müşteri güvenini sarsar.
    \item \textbf{Güvenlik açıkları:} Yetersiz test edilen API'ler veri ihlallerine zemin hazırlayabilir.
    \item \textbf{Operasyonel aksamalar:} Kritik işlemlerin başarısız olması iş süreçlerini durdurabilir.
\end{itemize}

Bu bağlamda, \textbf{projenin temel amacı} basitleştirilmiş bir e-ticaret sipariş ve ödeme sistemi için tasarlanan REST tabanlı web servisleri üzerinde yeniden kullanılabilir, modüler ve CI/CD uyumlu bir API test otomasyon çatısı geliştirmektir.

\textbf{Projenin hedefleri:}
\begin{enumerate}
    \item Ürün, sipariş, ödeme ve kullanıcı varlıklarını içeren örnek bir E-Ticaret REST API'si geliştirmek
    \item Fonksiyonel, negatif ve sınır değeri testlerini otomatikleştirmek
    \item Test paketini GitHub Actions ile sürekli entegrasyon ortamına entegre etmek
    \item Test sonuçlarını JUnit ve HTML formatlarda raporlamak
\end{enumerate}

\subsection{İlgili Çalışmalar}

RESTful API test otomasyonu alanında son yıllarda önemli akademik çalışmalar gerçekleştirilmiştir. Bu bölümde projemizin teorik temellerini oluşturan araştırmalar incelenmektedir.

\subsubsection{REST API Test Araştırmaları}

Golmohammadi ve arkadaşları (2023), REST API testine yönelik mevcut yaklaşımları kapsamlı bir şekilde inceleyen bir literatür taraması gerçekleştirmiştir [1]. Çalışmada 92 bilimsel makale analiz edilmiş ve test yaklaşımları dört ana kategoride sınıflandırılmıştır: kara-kutu testleri, beyaz-kutu testleri, model tabanlı testler ve sözleşme tabanlı testler. Araştırma, ağ iletişimi ve harici servis bağımlılıklarının API testindeki temel zorlukları oluşturduğunu ortaya koymaktadır.

Martin-Lopez ve arkadaşları (2021), OpenAPI spesifikasyonlarından otomatik test senaryoları üreten RESTest aracını geliştirmiştir [2]. Araç, fuzzing ve kısıtlama tabanlı test üretimini desteklemekte olup, hem çevrimdışı hem de çevrimiçi test senaryolarında başarıyla değerlendirilmiştir.

Viglianisi ve arkadaşları (2020), Swagger tanımlarından test istekleri üreten RESTTESTGEN aracını sunmuştur [3]. Araç, 87 gerçek dünya API'sinde başarıyla hatalar tespit etmiş olup, kaynak koda erişim gerektirmeden etkili kara-kutu testi yapılabileceğini göstermiştir.

\subsubsection{Test Tasarım Teknikleri}

Basili ve Selby (1987), fonksiyonel test tekniklerini deneysel olarak karşılaştıran öncül bir çalışma gerçekleştirmiştir [4]. Araştırma, eşdeğer bölgeleme ve sınır değer analizinin kontrol hatalarını tespit etmede diğer yöntemlerden daha etkili olduğunu ortaya koymuştur.

Reid (1997), havacılık sistemleri üzerinde yapılan deneylerle sınır değer analizinin (BVA) eşdeğer bölgeleme ve rastgele testten daha yüksek hata tespit oranı sağladığını kanıtlamıştır [5].

%%%%%%%%%%%%%%%%%%%%%%%%%%%%%%%%%%%%%%%%%%%%%%%%%%%%%%%%%%%%%%%%%%%%%%%%%%%%%%%
% BÖLÜM 2: PROBLEM TANIMI VE MEVCUT YÖNTEMLER (15 puan)
%%%%%%%%%%%%%%%%%%%%%%%%%%%%%%%%%%%%%%%%%%%%%%%%%%%%%%%%%%%%%%%%%%%%%%%%%%%%%%%
\section{Problem Tanımı ve Mevcut Yöntemler}

\subsection{Problem Tanımı}

E-ticaret sistemlerinde sipariş ve ödeme akışları, iş sürekliliği açısından kritik işlevlerdir. Sommerville (2016)'in belirttiği gibi, yazılım kalite güvencesi sistematik bir yaklaşım gerektirir ve hatalı yazılım ciddi sonuçlara yol açabilir [6]. Myers ve arkadaşları (2011), yazılım testinin temel amacının ``programda hata bulmak'' olduğunu vurgulamıştır [7].

Tipik bir e-ticaret alışveriş akışı aşağıdaki adımları içerir:
\begin{enumerate}
    \item \textbf{Kullanıcı Kaydı:} Yeni müşterilerin sisteme kaydolması
    \item \textbf{Kimlik Doğrulama:} Kayıtlı kullanıcıların sisteme giriş yapması
    \item \textbf{Ürün Listeleme:} Mevcut ürünlerin görüntülenmesi
    \item \textbf{Sipariş Oluşturma:} Sepete ürün ekleme ve sipariş verme
    \item \textbf{Ödeme İşlemi:} Siparişin ödenmesi
    \item \textbf{Sipariş Takibi:} Sipariş durumunun izlenmesi
\end{enumerate}

Her adımda test edilmesi gereken çeşitli iş kuralları bulunmaktadır:
\begin{itemize}
    \item \textbf{Kimlik doğrulama:} JWT (JSON Web Token) tabanlı oturum yönetimi
    \item \textbf{Yetkilendirme:} Rol bazlı erişim kontrolü (RBAC) -- müşteri ve admin rolleri
    \item \textbf{Stok kontrolü:} Yetersiz stokta sipariş engelleme
    \item \textbf{Sepet kuralları:} Minimum 50 TRY ve maksimum 5000 TRY tutar kısıtlamaları
    \item \textbf{Miktar sınırları:} Ürün başına 1-10 adet kısıtlaması
\end{itemize}

\subsection{Mevcut Yaklaşımlar ve Araçlar}

API test otomasyonu için çeşitli araçlar ve yaklaşımlar mevcuttur. Tablo 1'de popüler araçların karşılaştırması sunulmaktadır.

\begin{table}[H]
\centering
\caption{API Test Araçları Karşılaştırması}
\begin{tabular}{|l|c|c|c|c|}
\hline
\textbf{Araç} & \textbf{Programlama} & \textbf{CI/CD} & \textbf{Raporlama} & \textbf{Öğrenme} \\
\hline
Postman & Düşük kod & Orta & İyi & Kolay \\
REST Assured & Java & Yüksek & İyi & Orta \\
pytest + requests & Python & Yüksek & Esnek & Orta \\
Karate & DSL & Yüksek & İyi & Orta \\
\hline
\end{tabular}
\end{table}

\subsection{Mevcut Yöntemlerin Güçlü ve Zayıf Yönleri}

\textbf{Postman:}
\begin{itemize}
    \item Güçlü yönleri: Görsel arayüz, hızlı prototipleme, kolay kullanım
    \item Zayıf yönleri: Karmaşık test senaryoları için sınırlı, versiyon kontrolü zor
\end{itemize}

\textbf{REST Assured (Java):}
\begin{itemize}
    \item Güçlü yönleri: Güçlü DSL, kurumsal destek, geniş topluluk
    \item Zayıf yönleri: JVM bağımlılığı, verbose sözdizimi
\end{itemize}

\textbf{pytest + requests (Python):}
\begin{itemize}
    \item Güçlü yönleri: Esnek, geniş ekosistem, fixture desteği
    \item Zayıf yönleri: Başlangıç yapılandırması gerektirir
\end{itemize}

\subsection{Önerilen Çözüm}

Bu projede \textbf{pytest + requests} kombinasyonu tercih edilmiştir. Tercih nedenleri:
\begin{enumerate}
    \item Python'un okunabilir ve öğrenilmesi kolay sözdizimi
    \item pytest'in güçlü fixture mekanizması ile test bağımlılık yönetimi
    \item Modüler ve genişletilebilir mimari tasarımına uygunluk
    \item GitHub Actions ile kolay CI/CD entegrasyonu
    \item JUnit XML ve HTML formatlarda raporlama desteği
    \item Marker sistemi ile test kategorizasyonu
\end{enumerate}

%%%%%%%%%%%%%%%%%%%%%%%%%%%%%%%%%%%%%%%%%%%%%%%%%%%%%%%%%%%%%%%%%%%%%%%%%%%%%%%
% BÖLÜM 3: SİSTEM TASARIMI VE UYGULAMA GELİŞTİRME (40 puan)
%%%%%%%%%%%%%%%%%%%%%%%%%%%%%%%%%%%%%%%%%%%%%%%%%%%%%%%%%%%%%%%%%%%%%%%%%%%%%%%
\section{Sistem Tasarımı ve Uygulama Geliştirme}

\subsection{Sistem Mimarisi}

Proje iki ana bileşenden oluşmaktadır:
\begin{enumerate}
    \item \textbf{SUT (System Under Test):} FastAPI tabanlı E-Ticaret REST API
    \item \textbf{Test Otomasyon Çatısı:} pytest tabanlı modüler test framework
\end{enumerate}

\subsubsection{API Mimarisi}

FastAPI kullanılarak geliştirilen REST API, aşağıdaki özelliklere sahiptir:
\begin{itemize}
    \item Otomatik OpenAPI (Swagger) dokümantasyonu
    \item Pydantic tabanlı veri doğrulama
    \item JWT tabanlı kimlik doğrulama
    \item Bcrypt ile şifre hashleme
    \item Thread-safe bellek içi depolama
\end{itemize}

\subsubsection{Test Çatısı Mimarisi}

Test çatısı katmanlı bir mimari izlemektedir:

\textbf{1. Test Katmanı (test\_*.py):} Pytest test fonksiyonlarını içerir. Her test fonksiyonu bağımsız çalışır ve belirli bir senaryoyu test eder.

\textbf{2. Client Katmanı:} Domain-bazlı HTTP istemcileri sağlar:
\begin{itemize}
    \item AuthClient: Kayıt ve giriş işlemleri
    \item ProductClient: Ürün CRUD işlemleri
    \item OrderClient: Sipariş oluşturma ve yönetimi
    \item PaymentClient: Ödeme işlemleri
\end{itemize}

\textbf{3. Doğrulama Katmanı:} Response ve schema doğrulama yardımcıları:
\begin{itemize}
    \item assert\_status\_code(): HTTP durum kodu kontrolü
    \item validate\_order\_schema(): Sipariş yanıt şeması doğrulama
    \item validate\_error\_response\_schema(): Hata yanıt şeması doğrulama
\end{itemize}

\textbf{4. Veri Katmanı:} Test verisi üretim yardımcıları:
\begin{itemize}
    \item create\_order\_items(): Sipariş ögesi oluşturma
    \item generate\_email(): Benzersiz e-posta adresi üretimi
\end{itemize}

\subsection{İş Kuralları}

API'de uygulanan iş kuralları Tablo 2'de özetlenmiştir.

\begin{table}[H]
\centering
\caption{Uygulanan İş Kuralları}
\begin{tabular}{|l|c|l|}
\hline
\textbf{Kural} & \textbf{Değer} & \textbf{Açıklama} \\
\hline
MIN\_CART\_TOTAL & 50 TRY & Minimum sepet tutarı \\
MAX\_CART\_TOTAL & 5000 TRY & Maksimum sepet tutarı \\
MIN\_QTY & 1 & Ürün başına minimum miktar \\
MAX\_QTY & 10 & Ürün başına maksimum miktar \\
\hline
\end{tabular}
\end{table}

\subsection{API Endpoint'leri}

Geliştirilen API'de 11 endpoint bulunmaktadır. Tablo 3'te endpoint özeti verilmiştir.

\begin{table}[H]
\centering
\caption{API Endpoint Özeti}
\begin{tabular}{|l|l|l|}
\hline
\textbf{Grup} & \textbf{Endpoint} & \textbf{Açıklama} \\
\hline
Health & GET /health & Sistem sağlık kontrolü \\
\hline
Auth & POST /auth/register & Kullanıcı kaydı \\
Auth & POST /auth/login & Kullanıcı girişi (JWT) \\
\hline
Products & GET /products & Ürün listesi \\
Products & GET /products/\{id\} & Ürün detayı \\
Products & POST /products & Ürün oluşturma (admin) \\
\hline
Orders & POST /orders & Sipariş oluşturma \\
Orders & GET /orders/\{id\} & Sipariş detayı \\
Orders & POST /orders/\{id\}/cancel & Sipariş iptali \\
\hline
Payments & POST /payments & Ödeme oluşturma \\
Payments & GET /payments/\{id\} & Ödeme detayı \\
\hline
\end{tabular}
\end{table}

\subsection{Kullanılan Teknolojiler}

\begin{table}[H]
\centering
\caption{Teknoloji Yığını}
\begin{tabular}{|l|l|l|}
\hline
\textbf{Bileşen} & \textbf{Teknoloji} & \textbf{Gerekçe} \\
\hline
API Framework & FastAPI 0.109+ & Otomatik OpenAPI, yüksek performans \\
Test Framework & pytest 7.4+ & Fixture desteği, marker sistemi \\
HTTP Client & requests 2.31+ & Basit ve güvenilir \\
Kimlik Doğrulama & PyJWT + bcrypt & Endüstri standardı \\
Veri Doğrulama & Pydantic 2.0+ & Tip güvenliği \\
CI/CD & GitHub Actions & Ücretsiz, kolay entegrasyon \\
\hline
\end{tabular}
\end{table}

\subsection{Test Senaryoları}

Test senaryoları aşağıdaki kara-kutu test tasarım teknikleri kullanılarak tasarlanmıştır:

\subsubsection{Eşdeğer Bölgeleme}

Giriş alanı geçerli ve geçersiz değer sınıflarına ayrılmıştır. Örneğin:
\begin{itemize}
    \item Geçerli e-posta: user@example.com
    \item Geçersiz e-posta: invalid-email
\end{itemize}

\subsubsection{Sınır Değer Analizi}

Kritik sınır değerleri test edilmiştir:
\begin{itemize}
    \item Miktar: qty=1 (min), qty=10 (max), qty=11 (max+1)
    \item Sepet tutarı: 50 TRY (min), 5000 TRY (max)
\end{itemize}

\subsubsection{Negatif Testler}

Hatalı girişler ve yetkisiz erişim denemeleri:
\begin{itemize}
    \item Yanlış şifre ile giriş
    \item Geçersiz token ile istek
    \item Yetersiz stokta sipariş
\end{itemize}

\begin{table}[H]
\centering
\caption{Test Kategorileri ve Sayıları}
\begin{tabular}{|l|c|l|}
\hline
\textbf{Kategori} & \textbf{Test Sayısı} & \textbf{Kapsam} \\
\hline
Health & 1 & Sistem sağlık kontrolü \\
Auth & 5 & Kayıt, giriş, validasyon \\
Products & 4 & Listeleme, CRUD, yetkilendirme \\
Orders & 10 & Sipariş işlemleri, sınır değerler, iptal \\
Payments & 2 & Ödeme, yetkilendirme \\
Smoke (E2E) & 1 & Uçtan uca kritik akış \\
\hline
\textbf{Toplam} & \textbf{23} & \\
\hline
\end{tabular}
\end{table}

\subsection{Örnek Test Kodu}

Aşağıda sınır değer testi örneği verilmiştir:

\begin{lstlisting}[language=Python, caption=Sınır Değer Testi Örneği]
@pytest.mark.orders
def test_cart_total_at_minimum_boundary(
    product_client, order_client, admin_token, customer_token
):
    """Cart total exactly 50 TRY should succeed"""
    # Create product with price = 50 TRY
    p = product_client.create_product(
        admin_token, name="FiftyProduct", price=50.0, stock=10
    ).json()
    
    # Create order with qty=1, total = 50 TRY (minimum)
    r = order_client.create_order(
        customer_token, create_order_items(p["id"], qty=1)
    )
    
    assert_status_code(r, 201)
    assert r.json()["totalAmount"] == 50.0
\end{lstlisting}

\subsection{CI/CD Entegrasyonu}

GitHub Actions kullanılarak sürekli entegrasyon yapılandırılmıştır. Her push işleminde:
\begin{enumerate}
    \item Python ortamı kurulur
    \item Bağımlılıklar yüklenir
    \item API sunucusu başlatılır
    \item Tüm testler çalıştırılır
    \item JUnit ve HTML raporları oluşturulur
    \item Raporlar artifact olarak yüklenir
\end{enumerate}

\subsection{Test Çıktıları}

Tüm testler başarıyla geçmiştir:

\begin{verbatim}
============================= test session starts ==============================
platform darwin -- Python 3.14.0, pytest-9.0.2
collected 23 items

tests/test_auth.py .....                                                 [ 21%]
tests/test_health.py .                                                   [ 26%]
tests/test_orders.py ..........                                          [ 69%]
tests/test_payments.py ..                                                [ 78%]
tests/test_products.py ....                                              [ 95%]
tests/test_smoke.py .                                                    [100%]

============================= 23 passed in 10.64s ==============================
\end{verbatim}

%%%%%%%%%%%%%%%%%%%%%%%%%%%%%%%%%%%%%%%%%%%%%%%%%%%%%%%%%%%%%%%%%%%%%%%%%%%%%%%
% BÖLÜM 4: TARTIŞMA VE SONUÇ (20 puan)
%%%%%%%%%%%%%%%%%%%%%%%%%%%%%%%%%%%%%%%%%%%%%%%%%%%%%%%%%%%%%%%%%%%%%%%%%%%%%%%
\section{Tartışma ve Sonuç}

\subsection{Performans Değerlendirmesi}

Geliştirilen test otomasyon çatısı aşağıdaki metrikleri sağlamıştır:

\begin{table}[H]
\centering
\caption{Test Sonuç Metrikleri}
\begin{tabular}{|l|c|}
\hline
\textbf{Metrik} & \textbf{Değer} \\
\hline
Toplam Test Sayısı & 23 \\
Başarılı Test & 23 (100\%) \\
Başarısız Test & 0 \\
Toplam Çalışma Süresi & 10.64 saniye \\
Ortalama Test Süresi & 0.46 saniye \\
\hline
\end{tabular}
\end{table}

\subsection{Çalışmanın Önemi ve Katkıları}

Bu proje, yazılım kalite güvencesi alanına aşağıdaki katkıları sağlamaktadır:

\begin{enumerate}
    \item \textbf{Yeniden Kullanılabilir Çatı:} Domain-bazlı client sınıfları ve assertion helper'lar, farklı projelerde kolayca adapte edilebilir. Örneğin, yeni bir API endpoint'i test etmek için sadece ilgili client sınıfına yeni bir metod eklemek yeterlidir.
    
    \item \textbf{Test Tasarım Teknikleri Uygulaması:} Eşdeğer bölgeleme, sınır değer analizi ve negatif test teknikleri somut örneklerle gösterilmiştir. Bu teknikler, test kapsamını sistematik olarak artırmaya yardımcı olmaktadır.
    
    \item \textbf{CI/CD Entegrasyonu:} Testlerin her kod değişikliğinde otomatik çalıştırılması, regresyon hatalarının erken tespitini sağlamaktadır.
    
    \item \textbf{Dokümantasyon ve İzlenebilirlik:} OpenAPI sözleşmesi ve test kataloğu arasındaki eşleştirme, test kapsamının doğrulanabilmesini sağlamaktadır.
\end{enumerate}

\subsection{Kısıtlamalar}

\begin{itemize}
    \item \textbf{In-Memory Depolama:} Veritabanı yerine bellek içi depolama kullanılmıştır. Gerçek uygulamalarda veritabanı entegrasyonu gereklidir.
    
    \item \textbf{Performans Testi Yok:} Yük ve stres testleri proje kapsamında yer almamaktadır.
    
    \item \textbf{Güvenlik Testi Sınırlı:} OWASP Top 10 kapsamında güvenlik açığı taraması yapılmamıştır.
    
    \item \textbf{Tek Ortam:} Testler sadece lokal ve CI ortamında çalıştırılmıştır.
\end{itemize}

\subsection{Gelecek Çalışmalar}

Projenin gelecek sürümlerinde aşağıdaki iyileştirmeler planlanmaktadır:

\textbf{Altyapı İyileştirmeleri:}
\begin{itemize}
    \item Veritabanı entegrasyonu (PostgreSQL veya MongoDB) ile kalıcı veri depolama
    \item Konteyner tabanlı test ortamı (Docker ve Docker Compose ile)
    \item Çoklu ortam desteği (development, staging, production)
\end{itemize}

\textbf{Test Kapsamı Genişletme:}
\begin{itemize}
    \item Property-based testing (Hypothesis kütüphanesi ile rastgele test verisi üretimi)
    \item Performans ve yük testleri (Locust veya k6 araçlarıyla)
    \item Güvenlik testleri (OWASP ZAP entegrasyonu)
    \item API versiyonlama ve geriye uyumluluk testleri
\end{itemize}

\textbf{Test Kalitesi Artırma:}
\begin{itemize}
    \item Mock/stub kullanımı ile izole birim testleri
    \item Mutation testing ile test etkinliği ölçümü
    \item Test kapsam raporlaması (pytest-cov entegrasyonu)
    \item Paralel test çalıştırma (pytest-xdist ile)
\end{itemize}

\subsection{Sonuç}

Bu projede, e-ticaret sipariş ve ödeme servisleri için modüler ve yeniden kullanılabilir bir REST API test otomasyon çatısı başarıyla geliştirilmiştir. Geliştirilen çatı, eşdeğer bölgeleme, sınır değer analizi ve negatif test gibi kara-kutu test tasarım tekniklerini uygulayarak 23 test senaryosunu otomatikleştirmiştir. Tüm testler \%100 başarı oranıyla geçmiş olup, GitHub Actions ile CI/CD entegrasyonu sağlanmıştır.

Proje, yazılım kalite güvencesi alanında pratik bir uygulama sunmakta ve benzer API test projelerinde referans olarak kullanılabilecek bir altyapı sağlamaktadır.

%%%%%%%%%%%%%%%%%%%%%%%%%%%%%%%%%%%%%%%%%%%%%%%%%%%%%%%%%%%%%%%%%%%%%%%%%%%%%%%
% REFERANSLAR
%%%%%%%%%%%%%%%%%%%%%%%%%%%%%%%%%%%%%%%%%%%%%%%%%%%%%%%%%%%%%%%%%%%%%%%%%%%%%%%
\newpage
\section*{Referanslar}
\addcontentsline{toc}{section}{Referanslar}

\noindent [1] A. Golmohammadi, M. Zhang, and A. Arcuri, ``Testing RESTful APIs: A Survey,'' \textit{ACM Transactions on Software Engineering and Methodology}, vol. 33, no. 1, pp. 1--41, 2023. DOI: 10.1145/3617175

\vspace{0.3cm}
\noindent [2] A. Martin-Lopez, S. Segura, and A. Ruiz-Cortes, ``RESTest: Automated Black-Box Testing of RESTful Web APIs,'' in \textit{Proc. 30th ACM SIGSOFT International Symposium on Software Testing and Analysis (ISSTA '21)}, pp. 682--685, 2021. DOI: 10.1145/3460319.3469082

\vspace{0.3cm}
\noindent [3] E. Viglianisi, M. Dallago, and M. Ceccato, ``RESTTESTGEN: Automated Black-Box Testing of RESTful APIs,'' in \textit{Proc. IEEE 13th International Conference on Software Testing, Validation and Verification (ICST)}, pp. 142--152, 2020. DOI: 10.1109/ICST46399.2020.00024

\vspace{0.3cm}
\noindent [4] V. R. Basili and R. W. Selby, ``Comparing the Effectiveness of Software Testing Strategies,'' \textit{IEEE Transactions on Software Engineering}, vol. SE-13, no. 12, pp. 1278--1296, December 1987. DOI: 10.1109/TSE.1987.232881

\vspace{0.3cm}
\noindent [5] S. C. Reid, ``An Empirical Analysis of Equivalence Partitioning, Boundary Value Analysis and Random Testing,'' in \textit{Proc. Fourth International Software Metrics Symposium}, pp. 64--73, 1997. DOI: 10.1109/METRIC.1997.637166

\vspace{0.3cm}
\noindent [6] I. Sommerville, \textit{Software Engineering}, 10th ed. Pearson Education, 2016. ISBN: 978-0133943030

\vspace{0.3cm}
\noindent [7] G. J. Myers, C. Sandler, and T. Badgett, \textit{The Art of Software Testing}, 3rd ed. John Wiley \& Sons, 2011. ISBN: 978-1118031964

\vspace{0.3cm}
\noindent [8] FastAPI Documentation. [Online]. Available: \url{https://fastapi.tiangolo.com/} (Erişim: Aralık 2024)

\vspace{0.3cm}
\noindent [9] pytest Documentation. [Online]. Available: \url{https://docs.pytest.org/} (Erişim: Aralık 2024)

\end{document}
